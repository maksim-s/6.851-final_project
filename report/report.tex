\documentclass[12pt]{article}
\usepackage[utf8]{inputenc}

\usepackage{ifpdf}
\ifpdf
\usepackage[pdftex]{graphicx}
\else
\usepackage{graphicx}
\fi

\usepackage[margin=1.5in]{geometry}

\title{Tabulation Hashing Performance Benchmark}
\author{Maksim Stephenako \\ Yuzhi Zheng}

\date{May 2012}

\begin{document}

\setlength{\baselineskip}{1.25\baselineskip}

\ifpdf
\DeclareGraphicsExtensions{.pdf, .jpg, .tif}
\else
\DeclareGraphicsExtensions{.eps, .jpg}
\fi

\maketitle

\section{Introduction}
-everything uses hashing

- list languanges taht uses hashings for stuff
	and find other real world uses does databasee or something use hashing?
	
	-map reduce o.O?

-big o is important  but real performance is also important

- while multiplicative hashing might be one of the simpliest hash functions but tabulation hashing might be even better and have better performance as demonstrated by the paper and stuff

- we wahted to see how the other collisions resolution compare and  and the pros and cons of each


\section{Tabulation Hashing}
overall idea of tabulation hashing

make table

look stuff up etc
 
 - 3 independence
 
 -4 independence

- 5 independence
\section{Implementation}

overall view

about each function and mention the difference between them

talk about table sizes and how much space they take

problems we ran into

counting collisions instead of absolute time ( or maybe we can do both)


\section{Benchmark Results}

Compare pure hashing vs tabulation hashing

compare linear probing

compare quadratic probing

compare chaining

compare between the three

some analysis on memory access  and mention pros ad cons of each


\section{Conclusion}

summrize what we wanted to find out

what we did

and the results we found

\bibliographystyle{plain}
% Stuff will show up here if you use BibTeX
\bibliography{}

\end{document}

