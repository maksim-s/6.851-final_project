\documentclass[11pt]{article}
\usepackage[utf8]{inputenc}

\usepackage{ifpdf}
\ifpdf
\usepackage[pdftex]{graphicx}
\else
\usepackage{graphicx}
\fi

\usepackage[margin=1.25in]{geometry}

\title{Tabulation Hashing Performance Benchmark}
\author{Maksim Stephenako \\ Yuzhi Zheng}

\date{May 2012}

\begin{document}

\setlength{\baselineskip}{1\baselineskip}

\ifpdf
\DeclareGraphicsExtensions{.pdf, .jpg, .tif}
\else
\DeclareGraphicsExtensions{.eps, .jpg}
\fi

\maketitle

\section{Introduction}
% Yuzhi
Hashing is one of the most basic computer science concept. 
It allows elements to be reliably stored and 
retrieved from a limited number of slots, without dedicated slot of every possible 
variation of the element. While basic, hashing is used everywhere. Hashing is used in 
associative arrays, sometimes also known as dictionaries, in languages like 
PHP, Perl, and Python. Hashing can event be used for database indexing. 
Even lower level computer architectural components like processor caches 
use ideas from hashing to figure out which line to store value from a particular
memory address. Hashing can also be used to keep track of sets or make 
sure certain data representations are unique. Even the famous MapReduce
framework uses hashing to help shard inputs to be processed on different machines.

From a theoretical standpoint, hashing takes $O(1)$ time, which means it takes a constant
amount of time. That is essentially as fast as it gets. However, big-O notations can not
accurately depict the size of the constant factor. These constant factors sometimes
have a significant but real influence on the performance of any algorithm. 
Since hashing is used so often, it is important to keep that constant factor
as low as possible, and finding improvements whenever possible.

One of the most basic hashing function is the multiplicative hashing. 
Thorup and Zhang showed that a different type of hashing, tabulation hashing,
could potentially be a good alternative to the more basic multiplicative hashing 
in their paper from 2010. More specifically, they looked at the performance of tabulation hashing
used in conjunction with linear probing and found the performance to be competitive with
other hash functions on dense tables.

This report takes a closer look at tabulation hashing and it's performance 
against the basic multiplicative hashing. Instead of only looking at linear 
probing, we expanded our collision resolution techniques to quadratic probing 
and also chaining. We plan to do some benchmark testing as well as  
analyzing the possible pros and cons of each type of hash functions as 
well as the different collision resolutions.


\section{Tabulation Hashing}
% Maksim
overall idea of tabulation hashing

make table

look stuff up etc
 
- 3 independence
 
- 4 independence

- 5 independence
\section{Implementation}
% Yuzhi
overall view

about each function and mention the difference between them

talk about table sizes and how much space they take

problems we ran into

counting collisions instead of absolute time ( or maybe we can do both)


\section{Benchmark Results}
% Yuzhi

Compare pure hashing vs tabulation hashing

compare linear probing

compare quadratic probing

compare chaining

compare between the three

some analysis on memory access  and mention pros ad cons of each


\section{Conclusion}
% Maksim
summrize what we wanted to find out

what we did

and the results we found

\bibliographystyle{plain}
% Stuff will show up here if you use BibTeX
\bibliography{}

\end{document}

